\documentclass[12pt]{article}
\usepackage[french]{babel}
\usepackage[utf8]{inputenc}
\usepackage[T1]{fontenc}
\usepackage{xspace}
\usepackage{array}
\usepackage{color}
\usepackage{listings}
\usepackage{fancyvrb} %verbatim
\usepackage{tabularx}
\usepackage{eurosym}
\usepackage{mathrsfs}
\usepackage{stmaryrd}
\usepackage{amssymb}
\usepackage[?]{amsmath}
\usepackage{epsfig}
\usepackage{float}
\usepackage{wrapfig}
\usepackage{pifont}
\usepackage{enumerate}
\usepackage[np]{numprint} %\np{12345} donne 12 345 
\usepackage{multicol}
\newcommand{\second}{2\up{d}\xspace}
\newcommand{\seconde}{2\up{de}\xspace}
\newcommand{\R}{\mathbb R}
\newcommand{\Rp}{\R_+}
\newcommand{\Rpe}{\R_+^*}
\newcommand{\Rm}{\R_-}
\newcommand{\Rme}{\R_-^*}
\newcommand{\N}{\mathbb N}
\newcommand{\D}{\mathbb D}
\newcommand{\Q}{\mathbb Q}
\newcommand{\Z}{\mathbb Z}
\newcommand{\C}{\mathbb C}
\newcommand{\grs}{\mathfrak S}
\newcommand{\IN}[1]{\llbracket 1,#1\rrbracket}
\newcommand{\card}{\text{Card}\,}
\usepackage{mathrsfs}
\newcommand{\parties}{\mathscr P}
\renewcommand{\epsilon}{\varepsilon}
\newcommand{\rmd}{\text{d}}
\newcommand{\diff}{\mathrm D}
\newcommand{\Id}{{\rm Id}}
\newcommand{\e}{{\rm e}}
\newcommand{\I}{{\rm i}}
\newcommand{\J}{{\rm j}}
\newcommand{\ro}{\circ}
\newcommand{\exu}{\exists\,!\,}
\newcommand{\telq}{\,\, \mid \,\,}
\newcommand{\para}{\raisebox{0.1em}{\text{\footnotesize /\hspace{-0.1em}/}}}   
\newcommand{\vect}[1]{\overrightarrow{#1}}
\newcommand{\scal}[2]{\left(\, #1 \mid #2 \, \right)}
\newcommand{\ortho}[1]{{#1}^\perp}
\newcommand{\veci}{\vec{\text{\it \i}}}
\newcommand{\vecj}{\vec{\text{\it \j}}}
\newcommand{\rep}{$(O;\veci,\vecj,\vec{k})$\xspace}
\newcommand{\Oijk}{$(O, \veci,\vecj,\vec{k})$\xspace}
\newcommand{\rond}{repère orthonormal direct}
\newcommand{\bond}{base orthonormale directe}
\newcommand{\eq}{\Longleftrightarrow}
\newcommand{\implique}{\Longrightarrow}
\newcommand{\noneq}{\ \ \ /\hspace{-1.45em}\eq}
\newcommand{\tend}{\longrightarrow}
\newcommand{\egx}[2]{\underset{#1 \tend #2}=}
\newcommand{\asso}{\longmapsto}
\newcommand{\vers}{\longrightarrow}
\newcommand{\eqn}{~\underset{n \rightarrow \infty}{\sim}~}
\newcommand{\eqx}[2]{~\underset{#1 \rightarrow #2}{\sim}~}
\newcommand{\egn}{~\underset{n \rightarrow \infty}{=}~}
\renewcommand{\descriptionlabel}{\hspace{\labelsep}$\bullet$}
\renewcommand{\bar}{\overline}
\DeclareMathOperator{\ash}{{Argsh}}
\DeclareMathOperator{\cotan}{{cotan}}
\DeclareMathOperator{\ach}{{Argch}}
\DeclareMathOperator{\ath}{{Argth}}
\DeclareMathOperator{\sh}{{sh}}
\DeclareMathOperator{\ch}{{ch}}
%\DeclareMathOperator{\tanh}{{th}}
\DeclareMathOperator{\Mat}{{Mat}}
\DeclareMathOperator{\Vect}{{Vect}}
\DeclareMathOperator{\trace}{{tr}}
\newcommand{\tr}{{}^{\mathrm t}}
\newcommand{\divi}{~\big|~}
\newcommand{\ndivi}{~\not{\big|}~}
\newcommand{\et}{\wedge}
\newcommand{\ou}{\vee}
%\DeclareMathOperator{\trace}{{trace}}
\renewcommand{\det}{\operatorname{\text{dét}}}
%\DeclareMathOperator{\det}{{dét}}
\DeclareMathOperator{\grad}{{grad}}
%\DeclareMathOperator{\th}{th}
%\renewcommand{\tanh}{\mathop{\mathrm{th}}{}}
\renewcommand{\arcsin}{{\mathop{\mathrm{Arcsin}}}}
\renewcommand{\arccos}{{\mathop{\mathrm{Arccos}}}}
\renewcommand{\arctan}{{\mathop{\mathrm{Arctan}}}}
%\DeclareMathOperator{\arctan}{{Arctan}}
\renewcommand{\tanh}{{\mathop{\mathrm{th}}}}
\newcommand{\pgcd}{\mathop{\mathrm{pgcd}}}
\newcommand{\ppcm}{\mathop{\mathrm{ppcm}}}
\newcommand{\fonc}[4]{\left\{\begin{tabular}{ccc}$#1$ & $\vers$ & $#2$ \\
$#3$ & $\asso$ & $#4$ \end{tabular}\right.}
\renewcommand{\geq}{\geqslant}
\renewcommand{\leq}{\leqslant}
\renewcommand{\Re}{\text{\rm Re}}
\renewcommand{\Im}{\text{\rm Im}}
\renewcommand{\ker}{\mathop{\mathrm{Ker}}}
\newcommand{\Lin}{\mathcal L}
\newcommand{\GO}{\mathcal O}
\newcommand{\GSO}{\mathcal{SO}}
\newcommand{\GL}{\mathcal{GL}}
\renewcommand{\emptyset}{\varnothing}
\newcommand{\arc}[1]{\overset{\frown}{#1}}
\newcommand{\rg}{\mathop{\mathrm{rg}}}
\newcommand{\ds}{\displaystyle}
\newcommand{\co}[3]{\begin{pmatrix}#1 \\ #2 \\ #3\end{pmatrix}}
\newcommand{\demi}{\frac 1 2}

\newcommand{\exod}[1]{\addtocounter{cexo}{1} \vskip 1em 
\noindent{\bf \large Exercice \thecexo.\ #1}}
\newcounter{cact}
\renewcommand{\thecact}{\arabic{cact}}
\newcommand{\act}{\addtocounter{cact}{1} \vskip 1em \noindent{\bf Activité \thecact.\ }}
%\renewcommand{\labelitemi}{\m@th$\star$}
\newcommand{\limi}[2]{\underset{#1 \rightarrow #2}\lim}    
\newcommand{\nterme}[1]{{\em #1}}
\newcommand{\dif}[1]{#1}
\renewcommand{\FrenchLabelItem}{\textbullet} %bullet dans les items
%\renewcommand{\labelitemi}{\textbullet}
%\renewcommand{\labelitemii}{\textbullet}
%\renewcommand{\labelitemiii}{\textbullet}
%\renewcommand{\theenumi}{\alph{enumi}}
%\setenumerate[0]{label=\alph}
\usepackage{xr}
\usepackage{hyperref}
\usepackage{textcomp} %pour les ' dans lstlisting
\usepackage{listings}
%------------------ info
\lstset{
  language=Python,
  commentstyle=\color{red},
  upquote=true,
  literate={é}{{\'e}}1
           {ê}{{\^e}}1
           {Ã }{{\`a}}1
           {è}{{\`e}}1
           {î}{{\^i}}1
           {ï}{{\"i}}1
           {ë}{{\"e}}1
           {ù}{{\`u}}1,  
  columns=flexible,
  basicstyle=\ttfamily,
  keywordstyle=\color{blue},
  %identifierstyle=\color{green},
  stringstyle=\ttfamily,
  showstringspaces=false,
  %numbers=left, 
  %numberstyle=\tiny, 
  %stepnumber=1, 
  %numbersep=5pt,
  frame=L
}

 
\ifdefined\HCode
  \def\pgfsysdriver{pgfsys-dvisvgm4ht.def}
\fi 
\usepackage{pgf,tikz,pgfplots}
\usepackage{tkz-tab,tkz-fct}
\pgfplotsset{compat=1.16}
%\usepackage{mathrsfs}
\usetikzlibrary{arrows}

%-----------------------
\usepackage{fancybox}
\usepackage{fancyhdr}

\definecolor{gray}{HTML}{c8caca}
\definecolor{bleusad}{HTML}{0056B3}
\newcommand{\titreFiche}[3]{\centerline{\Ovalbox{\parbox[t]{0.7\textwidth}{
\vspace{-0.5em}\begin{center} \bf \large #1
\end{center}
\vspace{-0.5em}}}}

  \bigskip
\fancyfoot[L]{\includegraphics[height=1em]{/var/www/sacado/static/img/sacadoA1.png}\textcolor{bleusad}{\textsf{\ \raisebox{2pt}{ {{\footnotesize https://sacado.xyz > #3}}}}}}
\fancypagestyle{Page1}{\fancyhead{}\renewcommand{\headrulewidth}{0pt}}
\thispagestyle{Page1}
\fancyhead[L]{#1}
\fancyfoot[R]{\textsc{#2}}
\renewcommand{\footrulewidth}{0.4pt}
\pagestyle{fancy}
}

\newcounter{cexo}
\renewcommand{\thecexo}{\arabic{cexo}}
\newcommand{\exo}{\refstepcounter{cexo} \vskip 1.5em \noindent{\bf Exercice \thecexo.\ }}



\definecolor{bleu2}{rgb}{0,0.58,0.79}
\definecolor{violet}{rgb}{0.79,0.66,0.98}
\definecolor{gray}{rgb}{0.79,0.66,0.98}

\newcommand{\exercice}[1]{ % Exercice directement dans la page
\vspace{0.2cm}
\tikz\node[rounded corners=2pt, fill=bleu2]{\color{white}\textbf{#1}};
\vspace{0.1cm}
}

\newcommand{\sacado}[1]{ % Exercice directement dans la page
\vspace{0.2cm}
\tikz\node[rounded corners=2pt, fill=violet]{\color{white}\textbf{#1}};
}


\newcommand{\competence}[1]{
  \hfill\textcolor{bleusad}{#1}
}
\newcommand{\competences}[1]{
  {\bf Compétences ciblées : }
  \begin{enumerate}
   \item   #1
  \end{enumerate}

  \medskip
}

\newcommand{\savoir}[1]{
  {\bf Autre savoir-faire ciblé : } 
  #1
  \medskip
}

\newcommand{\savoirs}[1]{
  %  {\bf Autres savoir-faire ciblés : }
 \vspace{-1em}
  \textcolor{bleusad}{\it \begin{itemize}
      #1
  \end{itemize}
  \centerline{\rule[0.5em]{5cm}{0.5pt}}}
  \medskip
}


\newcommand{\entreexo}{ % entre les exos}
  \medskip
  \centerline{\hrule{5cm}{1pt}}
  \medskip
}

\setlength{\parindent}{0cm}
\usepackage[left=1cm,right=1cm,top=1cm,bottom=1.5cm]{geometry}
\usepackage{multicol}

%%%%% Pour faire des boites
\usepackage[tikz]{bclogo}
\usepackage{bclogo}
\usepackage{framed}
\usepackage[skins]{tcolorbox}
\tcbuselibrary{breakable}
\tcbuselibrary{skins}
\usetikzlibrary{quotes,babel,arrows.meta,shadows,decorations.pathmorphing,decorations.markings,patterns}
\usepackage{tikzpagenodes}
\usetikzlibrary{plotmarks}

\newcounter{cpt}

\newenvironment{GeneriqueT}[2]{%
\medskip \begin{tcolorbox}[widget,colback=white ,colframe=black,
title= \stepcounter{cpt}  #1  \thecpt \;: #2]}
{%
\end{tcolorbox}\par}



\newenvironment{defT}[1]{%
\medskip \begin{tcolorbox}[widget,colback=white ,colframe=black,
title= \stepcounter{cpt} Définition \thecpt \;: #1.]}
{%
\end{tcolorbox}\par}

\newenvironment{thT}[1]{%
\medskip \begin{tcolorbox}[widget,colback=white ,colframe=black,
title= \stepcounter{cpt} Théorème \thecpt\; : #1.]}
{%
\end{tcolorbox}\par}

\newenvironment{ppT}[1]{%
\medskip \begin{tcolorbox}[widget,colback=white ,colframe=black,
title= \stepcounter{cpt} Proposition \thecpt\; : #1.]}
{%
\end{tcolorbox}\par}

\newenvironment{exT}[1]{%
\medskip \begin{tcolorbox}[widget,colback=white ,colframe=black,
title= \stepcounter{cpt} Exemple \thecpt\; : #1.]}
{%
\end{tcolorbox}\par}

\newenvironment{mtT}[1]{%
\medskip \begin{tcolorbox}[widget,colback=white ,colframe=black,
title= \stepcounter{cpt} Méthode \thecpt \;: #1.]}
{%
\end{tcolorbox}\par}

\newenvironment{exoT}[1]{%
\medskip \begin{tcolorbox}[widget,colback=white ,colframe=black,
title= \stepcounter{cpt} Exercice \thecpt\; : #1.]}
{%
\end{tcolorbox}\par}

\newenvironment{sfT}[1]{%
\medskip \begin{tcolorbox}[widget,colback=white ,colframe=black,
title= \stepcounter{cpt} Savoir faire \thecpt \;: #1.]}
{%
\end{tcolorbox}\par}

\newenvironment{atT}[1]{%
\medskip \begin{tcolorbox}[widget,colback=white ,colframe=black,
title= \stepcounter{cpt} Attention \thecpt\; : #1.]}
{%
\end{tcolorbox}\par}



\newenvironment{voT}[1]{%
\medskip \begin{tcolorbox}[widget,colback=white ,colframe=black,
title= \stepcounter{cpt} Vocabulaire \thecpt\; : #1.]}
{%
\end{tcolorbox}\par}

\newenvironment{rmT}[1]{%
\medskip \begin{tcolorbox}[widget,colback=white ,colframe=black,
title= \stepcounter{cpt} Remarque \thecpt \;: #1.]}
{%
\end{tcolorbox}\par}

\newenvironment{mmT}[1]{%
\medskip \begin{tcolorbox}[widget,colback=white ,colframe=black,
title= \stepcounter{cpt} Mathématiciennes et mathématiciens \thecpt : #1.]}
{%
\end{tcolorbox}\par}


\newenvironment{soT}[1]{%
\medskip \begin{tcolorbox}[widget,colback=white ,colframe=black,
title= \stepcounter{cpt} Sommaire \thecpt\; : #1.]}
{%
\end{tcolorbox}\par}

\newenvironment{prT}[1]{%
\medskip \begin{tcolorbox}[widget,colback=white ,colframe=black,
title= \stepcounter{cpt} Preuve \thecpt \;: #1.]}
{%
\end{tcolorbox}\par}

\newenvironment{rgT}[1]{%
\medskip \begin{tcolorbox}[widget,colback=white ,colframe=black,
title= \stepcounter{cpt} Règle \thecpt\; : #1.]}
{%
\end{tcolorbox}\par}


\usepackage{multido}
\newcommand{\point}[1]{\vspace{0.1cm}\multido{}{#1}{ \dotfill \medskip \endgraf}}
\newcommand{\ligne}[1]{\vspace{0.1cm}\multido{}{#1}{ {\color{cqcqcq}\hrulefill} \medskip \endgraf}}\begin{document}
\titreFiche{BIB 1}{M. Demaria}\exo {\bf Nature du triangle }    \competence{Repr�senter. Calculer. }\vspace{0,2cm}\\On consid�re un triangle ABC dont les cot�s mesurent :$AB = 4\sqrt{3}$, $BC = 2\sqrt{12}$ et $CA= 4\sqrt{6}$. Quelle est la nature de ce triangle ?\vspace{0,2cm}\\\exo {\bf Factoriser des expressions. }    \competence{Calculer. }\vspace{0,2cm}\\Factorise les expressions suivantes :

\begin{enumerate}

\begin{minipage}{4cm}

\item $A=7x+7y$

\item $B=3x+6y$ 

\item $C=7a+6a$

\item $D=3\pi+5\pi$

\end{minipage}

\begin{minipage}{4cm}

\item $E=7x+21$

\item $F=x^2-2x$ 

\item $G=4a-8a^2$ 

\item $H=54-18b$ 

\end{minipage}

\begin{minipage}{4cm}

\item $I=-5b-25$ 

\item $J=3xy-9y^2$ 

\item $K=49x-21xy$ 

\item $L=-36a+63z$ 

\end{minipage}

\end{enumerate}\vspace{0,2cm}\\\exo {\bf Factoriser des expressions }    \competence{Calculer. }\vspace{0,2cm}\\Factorise les expressions suivantes :

\begin{enumerate}

\begin{minipage}{4cm}

\item $A=5x+10a$

\item $B=x+xy$ 

\item $C=7a+3a$

\item $D=\alpha+5\alpha$

\end{minipage}

\begin{minipage}{4cm}

\item $E=7x-21$

\item $F=-x^2+x$ 

\item $G=4a+16a^2$ 

\item $H=20-16b$ 

\end{minipage}

\begin{minipage}{4cm}

\item $I=-15b-25$ 

\item $J=12y-9y^2$ 

\item $K=30x-24$ 

\item $L=-36+27z^2$ 

\end{minipage}

\end{enumerate}\vspace{0,2cm}\\\exo {\bf Savoir factoriser une expression (rectangle). }    \competence{Chercher. Calculer. }\vspace{0,2cm}\\\begin{enumerate}

\item Un rectangle a pour longueur $L=16,5$ cm.

	\begin{enumerate}

		\item Quelle est la largeur $\ell$ ?

		\item Quelle est son aire pour $\ell = 7$cm  ?

	\end{enumerate}                                                          

\item Donne les mesures d'un autre rectangle de m�me p�rim�tre.

\item

	\begin{enumerate}

		\item La longueur peut-elle valoir 8 cm ? Justifie.

		\item La longueur peut-elle valoir 21 cm ? Justifie.

	\end{enumerate}                          



\item �cris une expression pour calculer la largeur $\ell$ en fonction de sa longueur $L$.

\item En voulant exprimer l'aire $A$ du rectangle en fonction de sa longueur $L$, des �l�ves ont donn� les r�ponses suivantes.



$$\text{Gael} : A=L\times 20 - L \quad \text{Hamid} : A=L\times (20 - L) \quad \text{Karen} : A=20L - L^2 $$



$$\text{Ines} : A=2\times L + 2 \times (20 - L) \quad \text{Sara} : A=L\times 20 - 2\times L \quad \text{Josua} : A= L^2 - 20\times L$$



Parmi ces expressions, lesquelles sont fausses ? Y a-t-il plusieurs bonnes r�ponses ?

\end{enumerate}\vspace{0,2cm}\\\exo {\bf Savoir factoriser une expression (demi cercle). }    \competence{}\vspace{0,2cm}\\\begin{minipage}{8cm}

On donne le dessin ci-contre.



Le demi cercle bleu a un rayon de $R_1$ et le demi cercle rouge a un rayon de $R_2$.



\begin{enumerate}

\item Exprime la longueur $\ell_1$ de l'arc $\wideparen{AC}$.

\item Exprime la longueur $\ell_2$ de l'arc $\wideparen{BC}$.

\item Exprime la longueur $\ell$ de l'arc $\wideparen{AB}$.

\item Compare alors $\ell$ et $\ell_1 + \ell_2$.

\end{enumerate}

\end{minipage}

\begin{minipage}{8cm}



\definecolor{qqwuqq}{rgb}{0.,0.39215686274509803,0.}

\definecolor{ffqqqq}{rgb}{1.,0.,0.}

\definecolor{qqqqff}{rgb}{0.,0.,1.}

\definecolor{sqsqsq}{rgb}{0.12549019607843137,0.12549019607843137,0.12549019607843137}

\begin{tikzpicture}[line cap=round,line join=round,>=triangle 45,x=1.0cm,y=1.0cm]

\clip(1.42,-3.04) rectangle (8.76,1.54);

\draw (2.,-2.)-- (8.,-2.);

\draw [shift={(4.,-2.)},color=qqqqff]  plot[domain=0.:3.141592653589793,variable=\t]({1.*2.*cos(\t r)+0.*2.*sin(\t r)},{0.*2.*cos(\t r)+1.*2.*sin(\t r)});

\draw [shift={(7.,-2.)},color=ffqqqq]  plot[domain=0.:3.141592653589793,variable=\t]({1.*1.*cos(\t r)+0.*1.*sin(\t r)},{0.*1.*cos(\t r)+1.*1.*sin(\t r)});

\draw [shift={(5.,-2.)},color=qqwuqq]  plot[domain=0.:3.141592653589793,variable=\t]({1.*3.*cos(\t r)+0.*3.*sin(\t r)},{0.*3.*cos(\t r)+1.*3.*sin(\t r)});

\begin{scriptsize}

\draw [color=sqsqsq] (2.,-2.)-- ++(-2.5pt,0 pt) -- ++(5.0pt,0 pt) ++(-2.5pt,-2.5pt) -- ++(0 pt,5.0pt);

\draw[color=sqsqsq] (2.06,-2.34) node {$A$};

\draw [color=sqsqsq] (8.,-2.)-- ++(-2.5pt,0 pt) -- ++(5.0pt,0 pt) ++(-2.5pt,-2.5pt) -- ++(0 pt,5.0pt);

\draw[color=sqsqsq] (8.24,-2.24) node {$B$};

\draw [color=sqsqsq] (6.,-2.)-- ++(-2.5pt,0 pt) -- ++(5.0pt,0 pt) ++(-2.5pt,-2.5pt) -- ++(0 pt,5.0pt);

\draw[color=sqsqsq] (6.06,-2.22) node {$C$};

\end{scriptsize}

\end{tikzpicture}



\end{minipage}\vspace{0,2cm}\\\exo {\bf  }    \competence{Repr�senter. }\vspace{0,2cm}\\Tracer la figure suivante en utilisant le logiciel Scratch, puis en tracer son sym�trique par rapport � O.



\definecolor{ffqqqq}{rgb}{1.,0.,0.}

\definecolor{cqcqcq}{rgb}{0.7529411764705882,0.7529411764705882,0.7529411764705882}

\begin{tikzpicture}[line cap=round,line join=round,>=triangle 45,x=0.8849557522123898cm,y=1.0cm]

\draw [color=cqcqcq,, xstep=0.8849557522123898cm,ystep=1.0cm] (-5.98,-4.04) grid (3.06,4.);

\clip(-5.98,-4.04) rectangle (3.06,4.);

\draw [line width=2.pt] (2.,0.)-- (-5.,3.);

\draw [line width=2.pt] (-5.,3.)-- (-3.,-1.);

\draw [line width=2.pt] (-3.,-1.)-- (-5.,-3.);

\draw [line width=2.pt] (-5.,-3.)-- (2.,0.);

\begin{scriptsize}

\draw [fill=ffqqqq] (0.,0.) circle (2.5pt);

\draw[color=ffqqqq] (0.14,0.37) node {$O$};

\end{scriptsize}

\end{tikzpicture}\vspace{0,2cm}\\\exo {\bf Dessiner un triangle �quilat�ral - Dessiner un rectangle. }    \competence{Calculer. }\vspace{0,2cm}\\\begin{enumerate}

\item \textbf{Dessiner un triangle �quilat�ral}

\begin{enumerate}

\item �crire un algorithme qui dessine un triangle �quilat�ral de c�t� de longueur 10 px.

\item Utiliser Scratch pour dessiner ce triangle. R�duire � 10\% la taille de Scratch.

\end{enumerate}



\item \textbf{Dessiner un rectangle}

\begin{enumerate}

\item �crire un algorithme qui dessine un rectangle de longueur 100 px et de largeur 50 px.

\item Utiliser Scratch pour dessiner ce rectangle. R�duire � 10\% la taille de Scratch.

\end{enumerate}

\end{enumerate}\vspace{0,2cm}\\\exo {\bf Calculer la somme de deux nombres. }    \competence{Calculer. }\vspace{0,2cm}\\\begin{enumerate}

\item \begin{enumerate}

		\item �crire un algorithme qui demande 2 nombres $x$ et $y$ puis calcule $x^2-y^2$.

		\item �crire un programme qui traduit cet algorithme. Utiliser le lutin Adrian pour demander les valeurs et donner le r�sultat.

		\end{enumerate}

\item \begin{enumerate}

		\item �crire un algorithme qui demande 2 nombres $x$ et $y$ puis calcule $(x-y)(x+y)$.

		\item �crire un programme qui traduit cet algorithme. Utiliser le lutin Alex pour demander les valeurs et donner le r�sultat. 

		\end{enumerate}

\item Comparer les 2 r�sultats �nonc�s par Adrian et Alex.

\end{enumerate}\vspace{0,2cm}\\\exo {\bf Donner un nom � une variable. }    \competence{Calculer. }\vspace{0,2cm}\\Voici un algorithme.



\begin{algobox}

\Variables

\Ligne x EST\_DU\_TYPE NOMBRE

\DebutAlgo

\Ligne LIRE x

\Ligne x PREND\_LA\_VALEUR x + 5

\Ligne x PREND\_LA\_VALEUR x/4

\Ligne AFFICHER x

\FinAlgo

\end{algobox}



\begin{enumerate}

\item Donner le nom de la variable.

\item Quelle est la valeur de sortie pour $x=4$.

\item Quelle est la valeur de sortie pour $x=-1$.

\item Quelle formule aurait-on pu �crire en une seule ligne pour remplacer les lignes 5 et 6 ?

\item Modifier alors cet algorithme avec la formule de la question pr�c�dente.

\end{enumerate}\vspace{0,2cm}\\\end{document}