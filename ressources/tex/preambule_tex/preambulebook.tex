\documentclass[dvipsnames,french,10pt]{book}
%----------------------------------------------------------------------------------------
%   Géometrie de la page
%----------------------------------------------------------------------------------------
\usepackage[
paperheight=29.5cm, %hauteur du papier
paperwidth=21cm, %largeur du papier
left=.8cm, %marge de gauche
right=.8cm, %marge de droite
top=1.35cm, %marge du haut
bottom=1cm, %marge du bas
%marginparsep=0pt, %distance entre le texte et les notes de marges 
headheight=20.60pt, %hauteur du header
%showframe, %permet d'afficher le cadre défini ci-dessus
%bindingoffset=1cm %permet d'ajouter le décalage dû au reliage
]{geometry } %Redéfinition de la taille des pages
\raggedbottom


%----------------------------------------------------------------------------------------
%   Generals
%----------------------------------------------------------------------------------------
%\usepackage{fourier} %!! A changer plus tard !!
\usepackage[scaled]{uarial}
\renewcommand*\familydefault{\sfdefault} %% Only if the base font of the document is to be sans serif
\usepackage{frcursive}
\usepackage[T1]{fontenc} %Accents handling
\usepackage[utf8]{inputenc} % Use UTF-8 encoding
%\usepackage{microtype} % Slightly tweak font spacing for aesthetics
\usepackage[english, francais]{babel} % Language hyphenation and typographical rules
\usepackage{marginnote}


%----------------------------------------------------------------------------------------
%   Graphics
%----------------------------------------------------------------------------------------
\usepackage{xcolor}
\usepackage{graphicx, multicol} % Enhanced support for graphics
\graphicspath{FIG/}
\usepackage{wrapfig}
\usepackage{colortbl}
\usepackage[framemethod=tikz]{mdframed}
%\usepackage{xsavebox}
% Il faudrait utiliser xsavebox à l'avenir pour réduire la taille du pdf

%----------------------------------------------------------------------------------------
%   Other packages
%----------------------------------------------------------------------------------------
\usepackage{hyperref}
\hypersetup{
  colorlinks=true, %colorise les liens
  breaklinks=true, %permet le retour à la ligne dans les liens trop longs
  urlcolor= sacado_violet,  %couleur des hyperliens et des QR codes
  linkcolor= sacado_violet, %couleur des liens internes
  plainpages=false  %pour palier à "Bookmark problems can occur when you have duplicate page numbers, for example, if you have a page i and a page 1."
}
\usepackage{tabularx}
\newcolumntype{M}[1]{>{\arraybackslash}m{#1}} %Defines a scalable column type in tabular
\usepackage{booktabs} % Enhances quality of tables
\usepackage{diagbox} % barre en diagonale dans un tableau
\usepackage{multicol}
\usepackage[explicit]{titlesec}
\usepackage{xr}
\usepackage{xspace}
\usepackage{array}
\usepackage{listings}
\usepackage{fancyvrb} %verbatim
\usepackage{stmaryrd}
\usepackage{float}



% Python style for highlighting
\lstdefinestyle{mystyle}{
    backgroundcolor=\color{white},   
    commentstyle=\color{sacado_green},
    keywordstyle=\color{sacado_red},
    numberstyle=\tiny\color{sacado_orange},
    stringstyle=\color{sacado_blue},
    basicstyle=\ttfamily\footnotesize,
    breakatwhitespace=false,         
    breaklines=true,                 
    captionpos=b,                    
    keepspaces=false,                 
    numbers=left,                    
    numbersep=5pt,                  
    showspaces=false,                
    showstringspaces=false,
    showtabs=false,                  
    tabsize=4
}

\lstset{style=mystyle}

%----------------------------------------------------------------------------------------
%   Headers and footers
%----------------------------------------------------------------------------------------

\pagestyle{empty}
\usepackage{fancyhdr}
\pagestyle{fancy}
\renewcommand{\headrulewidth}{0pt} % pas de filet sous le header

%----------------------------------------------------------------------------------------
%   Mathematics packages
%----------------------------------------------------------------------------------------
\usepackage{amsthm, amsmath, amssymb, mathrsfs} % Mathematical typesetting
\usepackage{marvosym, wasysym} % More symbols
\usepackage[makeroom]{cancel}
\usepackage{xlop}
\usepackage{pgf,tikz,pgfplots}
\pgfplotsset{compat=1.16}
\usepackage{pgf-pie}
\usetikzlibrary{positioning}
\usetikzlibrary{arrows}
\usepackage{pst-plot,pst-tree,pst-func,pstricks-add,pst-node,pst-text}
%\usepackage{units}
\usepackage{nicefrac}
\usepackage[np]{numprint} %Séparation milliers dans un nombre \np{12345} donne 12 345
\usepackage{multido}
\newcommand{\RNum}[1]{\uppercase\expandafter{\romannumeral #1\relax}}

%----------------------------------------------------------------------------------------
%   New text commands
%----------------------------------------------------------------------------------------
\usepackage{calc}
\usepackage{boites}
 \renewcommand{\arraystretch}{1.6}

%%%%% Pour les imports.
\usepackage{import}

%%%%% Pour faire des boites
\usepackage[tikz]{bclogo}
\usepackage{bclogo}
\usepackage{framed}
\usepackage[skins]{tcolorbox}
\tcbuselibrary{breakable}
\tcbuselibrary{skins}
\usetikzlibrary{quotes,babel,arrows.meta,shadows,decorations.pathmorphing,decorations.markings,patterns}
\usepackage{tikzpagenodes}
\usetikzlibrary{plotmarks}



%%%%% Pour une double minipage
\newcommand{\mini}[4]{
\noindent
\begin{minipage}[c]{#1}
#2
\end{minipage}
\hfill
\noindent
\begin{minipage}[c]{#3}
#4
\end{minipage}
}


\usepackage{enumitem}
\newlist{todolist}{itemize}{2} %Pour faire des QCM
\setlist[todolist]{label=$\square$} %Pour faire des QCM \begin{todolist} instead of itemize
\renewcommand{\FrenchLabelItem}{\textbullet} %bullet dans les items


%----------------------------------------------------------------------------------------
%   Définition de couleurs pour ...
%----------------------------------------------------------------------------------------

%GEOGEBRA

\definecolor{zzttqq}{rgb}{0.6,0.2,0.} %rouge des polygones
\definecolor{qqqqff}{rgb}{0.,0.,1.}
\definecolor{xdxdff}{rgb}{0.49019607843137253,0.49019607843137253,1.}%bleu
\definecolor{qqwuqq}{rgb}{0.,0.39215686274509803,0.} %vert des angles
\definecolor{ffqqqq}{rgb}{1.,0.,0.} %rouge vif
\definecolor{uuuuuu}{rgb}{0.26666666666666666,0.26666666666666666,0.26666666666666666}
\definecolor{qqzzqq}{rgb}{0.,0.6,0.}
\definecolor{cqcqcq}{rgb}{0.7529411764705882,0.7529411764705882,0.7529411764705882} %gris
\definecolor{qqffqq}{rgb}{0.,1.,0.}
\definecolor{ffdxqq}{rgb}{1.,0.8431372549019608,0.}
\definecolor{ffffff}{rgb}{1.,1.,1.}
\definecolor{ududff}{rgb}{0.30196078431372547,0.30196078431372547,1.}
\definecolor{ffqqff}{rgb}{1.,0.,1.}
\definecolor{ffxfqq}{rgb}{1,0.4980392156862745,0}
\definecolor{ffffqq}{rgb}{1,1,0}
\definecolor{qqttzz}{rgb}{0,0.2,0.6}
\definecolor{qqccqq}{rgb}{0,0.8,0}
\definecolor{qqzzff}{rgb}{0,0.6,1}
\definecolor{qqwwzz}{rgb}{0,0.4,0.6}
\definecolor{eqeqeq}{rgb}{0.8784313725490196,0.8784313725490196,0.8784313725490196}

%SACADO

\definecolor{fond}{HTML}{542885}  %couleur des entetes etc.  violet sacado
\definecolor{sacado_purple}{HTML}{542885} %% Violet foncé Sacado
\definecolor{sacado_violet}{HTML}{ead7ff} %% Violet clair Sacado
\definecolor{texte}{HTML}{542885} % couleur du texte des entetes etc.
\definecolor{sacado_blue}{HTML}{54c3e8} %% Bleu Sacado
\definecolor{sacado_blue_light}{HTML}{e5f4fc} %% Bleu Sacado
\definecolor{sacado_green}{HTML}{57b13c} %% Vert Sacado
\definecolor{sacado_green_light}{HTML}{e5fce8} %% Vert Sacado foncé
\definecolor{sacado_yellow}{HTML}{ffdc7a} %% Jaune Sacado
\definecolor{sacado_yellow_light}{HTML}{fcfce5} %% Jaune Sacado
\definecolor{sacado_orange}{HTML}{e8c212} %% Orange Sacado
\definecolor{sacado_pink}{HTML}{ea4d88} %% Rouge Sacado
\definecolor{sacado_pink_light}{HTML}{fce5ee} %% Rouge Sacado
%BOITES 

\definecolor{bleu1}{rgb}{0.54,0.79,0.95} %% Bleu
\definecolor{sapgreen}{rgb}{0.4, 0.49, 0}
\definecolor{dvzfxr}{rgb}{0.7,0.4,0.}
\definecolor{beamer}{rgb}{0.5176470588235295,0.49019607843137253,0.32941176470588235} % couleur beamer
\definecolor{preuveRbeamer}{rgb}{0.8,0.4,0}
\definecolor{sectioncolor}{rgb}{0.24,0.21,0.44}
\definecolor{subsectioncolor}{rgb}{0.1,0.21,0.61}
\definecolor{subsubsectioncolor}{rgb}{0.1,0.21,0.61}
\definecolor{info}{rgb}{0.82,0.62,0}
\definecolor{bleu2}{rgb}{0.38,0.56,0.68}
\definecolor{bleu3}{rgb}{0.24,0.34,0.40}
\definecolor{bleu4}{rgb}{0.12,0.20,0.25}
\definecolor{vert}{rgb}{0.21,0.33,0}
\definecolor{vertS}{rgb}{0.05,0.6,0.42}
\definecolor{red}{rgb}{0.78,0,0}
\definecolor{color5}{rgb}{0,0.4,0.58}
\definecolor{eduscol4B}{rgb}{0.19,0.53,0.64}
\definecolor{eduscol4P}{rgb}{0.62,0.12,0.39}
\definecolor{ill_frame}{HTML}{F0F0F0} %Boite illustration contour
\definecolor{ill_back}{HTML}{F7F7F7}  %Boite illustration background
\definecolor{ill_title}{HTML}{AAAAAA} %Boite illustration titre



\usepackage[T1]{fontenc}
\usepackage{xspace}
\usepackage{array}
\usepackage{color}
\usepackage{listings}
\usepackage{fancyvrb} %verbatim
\usepackage{tabularx}
\usepackage{eurosym}
\usepackage{mathrsfs}
\usepackage{stmaryrd}
\usepackage{amssymb}
\usepackage{epsfig}
\usepackage{float}
\usepackage{wrapfig}
\usepackage{pifont}
 
\usepackage[np]{numprint} %\np{12345} donne 12 345 
\usepackage{multicol}
\newcommand{\second}{2\up{d}\xspace}
\newcommand{\seconde}{2\up{de}\xspace}
\newcommand{\R}{\mathbb R}
\newcommand{\Rp}{\R_+}
\newcommand{\Rpe}{\R_+^*}
\newcommand{\Rm}{\R_-}
\newcommand{\Rme}{\R_-^*}
\newcommand{\N}{\mathbb N}
\newcommand{\D}{\mathbb D}
\newcommand{\Q}{\mathbb Q}
\newcommand{\Z}{\mathbb Z}
\newcommand{\C}{\mathbb C}
\newcommand{\grs}{\mathfrak S}
\newcommand{\IN}[1]{\llbracket 1,#1\rrbracket}
\newcommand{\card}{\text{Card}\,}
\usepackage{mathrsfs}
\newcommand{\parties}{\mathscr P}
\renewcommand{\epsilon}{\varepsilon}
\newcommand{\rmd}{\text{d}}
\newcommand{\diff}{\mathrm D}
\newcommand{\Id}{{\rm Id}}
\newcommand{\e}{{\rm e}}
\newcommand{\I}{{\rm i}}
\newcommand{\J}{{\rm j}}
\newcommand{\ro}{\circ}
\newcommand{\exu}{\exists\,!\,}
\newcommand{\telq}{\,\, \mid \,\,}
\newcommand{\para}{\raisebox{0.1em}{\text{\footnotesize /\hspace{-0.1em}/}}}   
\newcommand{\vect}[1]{\overrightarrow{#1}}
\newcommand{\scal}[2]{\left(\, #1 \mid #2 \, \right)}
\newcommand{\ortho}[1]{{#1}^\perp}
\newcommand{\veci}{\vec{\text{\it \i}}}
\newcommand{\vecj}{\vec{\text{\it \j}}}
\newcommand{\rep}{$(O;\veci,\vecj,\vec{k})$\xspace}
\newcommand{\Oijk}{$(O, \veci,\vecj,\vec{k})$\xspace}
\newcommand{\rond}{repère orthonormal direct}
\newcommand{\bond}{base orthonormale directe}
\newcommand{\eq}{\Longleftrightarrow}
\newcommand{\implique}{\Longrightarrow}
\newcommand{\noneq}{\ \ \ /\hspace{-1.45em}\eq}
\newcommand{\tend}{\longrightarrow}
\newcommand{\egx}[2]{\underset{#1 \tend #2}=}
\newcommand{\asso}{\longmapsto}
\newcommand{\vers}{\longrightarrow}
\newcommand{\eqn}{~\underset{n \rightarrow \infty}{\sim}~}
\newcommand{\eqx}[2]{~\underset{#1 \rightarrow #2}{\sim}~}
\newcommand{\egn}{~\underset{n \rightarrow \infty}{=}~}
\renewcommand{\descriptionlabel}{\hspace{\labelsep}$\bullet$}
\renewcommand{\bar}{\overline}
\DeclareMathOperator{\ash}{{Argsh}}
\DeclareMathOperator{\cotan}{{cotan}}
\DeclareMathOperator{\ach}{{Argch}}
\DeclareMathOperator{\ath}{{Argth}}
\DeclareMathOperator{\sh}{{sh}}
\DeclareMathOperator{\ch}{{ch}}
%\DeclareMathOperator{\tanh}{{th}}
\DeclareMathOperator{\Mat}{{Mat}}
\DeclareMathOperator{\Vect}{{Vect}}
\DeclareMathOperator{\trace}{{tr}}
\newcommand{\tr}{{}^{\mathrm t}}
\newcommand{\divi}{~\big|~}
\newcommand{\ndivi}{~\not{\big|}~}
\newcommand{\et}{\wedge}
\newcommand{\ou}{\vee}
%\DeclareMathOperator{\trace}{{trace}}
\renewcommand{\det}{\operatorname{\text{dét}}}
%\DeclareMathOperator{\det}{{dét}}
\DeclareMathOperator{\grad}{{grad}}
%\DeclareMathOperator{\th}{th}
%\renewcommand{\tanh}{\mathop{\mathrm{th}}{}}
\renewcommand{\arcsin}{{\mathop{\mathrm{Arcsin}}}}
\renewcommand{\arccos}{{\mathop{\mathrm{Arccos}}}}
\renewcommand{\arctan}{{\mathop{\mathrm{Arctan}}}}
%\DeclareMathOperator{\arctan}{{Arctan}}
\renewcommand{\tanh}{{\mathop{\mathrm{th}}}}
\newcommand{\pgcd}{\mathop{\mathrm{pgcd}}}
\newcommand{\ppcm}{\mathop{\mathrm{ppcm}}}
\newcommand{\fonc}[4]{\left\{\begin{tabular}{ccc}$#1$ & $\vers$ & $#2$ \\
$#3$ & $\asso$ & $#4$ \end{tabular}\right.}
\renewcommand{\geq}{\geqslant}
\renewcommand{\leq}{\leqslant}
\renewcommand{\Re}{\text{\rm Re}}
\renewcommand{\Im}{\text{\rm Im}}
\renewcommand{\ker}{\mathop{\mathrm{Ker}}}
\newcommand{\Lin}{\mathcal L}
\newcommand{\GO}{\mathcal O}
\newcommand{\GSO}{\mathcal{SO}}
\newcommand{\GL}{\mathcal{GL}}
\renewcommand{\emptyset}{\varnothing}
\newcommand{\rg}{\mathop{\mathrm{rg}}}
\newcommand{\ds}{\displaystyle}
\newcommand{\co}[3]{\begin{pmatrix}#1 \\ #2 \\ #3\end{pmatrix}}
\newcommand{\demi}{\frac 1 2}

\newcommand{\exod}[1]{\addtocounter{cexo}{1} \vskip 1em 
\noindent{\bf \large Exercice \thecexo.\ #1}}
\newcounter{cact}
\renewcommand{\thecact}{\arabic{cact}}
\newcommand{\act}{\addtocounter{cact}{1} \vskip 1em \noindent{\bf Activité \thecact.\ }}
%\renewcommand{\labelitemi}{\m@th$\star$}
\newcommand{\limi}[2]{\underset{#1 \rightarrow #2}\lim}    
\newcommand{\nterme}[1]{{\em #1}}
\newcommand{\dif}[1]{#1}
\renewcommand{\FrenchLabelItem}{\textbullet} %bullet dans les items
%\renewcommand{\labelitemi}{\textbullet}
%\renewcommand{\labelitemii}{\textbullet}
%\renewcommand{\labelitemiii}{\textbullet}
%\renewcommand{\theenumi}{\alph{enumi}}
%\setenumerate[0]{label=\alph}
\usepackage{xr}
\usepackage{hyperref}
\usepackage{textcomp} %pour les ' dans lstlisting
\usepackage{listings}
%------------------ info
\lstset{
  language=Python,
  commentstyle=\color{red},
  upquote=true,
  literate={é}{{\'e}}1
           {ê}{{\^e}}1
           {Ã }{{\`a}}1
           {è}{{\`e}}1
           {î}{{\^i}}1
           {ï}{{\"i}}1
           {ë}{{\"e}}1
           {ù}{{\`u}}1,  
  columns=flexible,
  basicstyle=\ttfamily,
  keywordstyle=\color{blue},
  %identifierstyle=\color{green},
  stringstyle=\ttfamily,
  showstringspaces=false,
  %numbers=left, 
  %numberstyle=\tiny, 
  %stepnumber=1, 
  %numbersep=5pt,
  frame=L
}

 
\ifdefined\HCode
  \def\pgfsysdriver{pgfsys-dvisvgm4ht.def}
\fi 
\usepackage{pgf,tikz,pgfplots}
\usepackage{tkz-tab,tkz-fct}
\pgfplotsset{compat=1.16}
%\usepackage{mathrsfs}
\usetikzlibrary{arrows}

%-----------------------
\usepackage{fancybox}
\usepackage{fancyhdr}

\definecolor{gray}{HTML}{c8caca}
\definecolor{bleusad}{HTML}{0056B3}
\newcommand{\titreFiche}[3]{\centerline{\Ovalbox{\parbox[t]{0.7\textwidth}{
\vspace{-0.5em}\begin{center} \bf \large #1
\end{center}
\vspace{-0.5em}}}}

  \bigskip
\fancyfoot[L]{\includegraphics[height=1em]{/var/www/sacado/static/img/sacadoA1.png}\textcolor{bleusad}{\textsf{\ \raisebox{2pt}{ {{\footnotesize https://sacado.xyz > #3}}}}}}
\fancypagestyle{Page1}{\fancyhead{}\renewcommand{\headrulewidth}{0pt}}
\thispagestyle{Page1}
\fancyhead[L]{#1}
\fancyfoot[R]{\textsc{#2}}
\renewcommand{\footrulewidth}{0.4pt}
\pagestyle{fancy}
}

\newcounter{cexo}
\renewcommand{\thecexo}{\arabic{cexo}}
\newcommand{\exo}{\refstepcounter{cexo} \vskip 1.5em \noindent{\bf Exercice \thecexo.\ }}



\definecolor{bleu2}{rgb}{0,0.58,0.79}
\definecolor{violet}{rgb}{0.79,0.66,0.98}
\definecolor{gray}{rgb}{0.79,0.66,0.98}

\newcommand{\exercice}[1]{ % Exercice directement dans la page
\vspace{0.2cm}
\tikz\node[rounded corners=2pt, fill=bleu2]{\color{white}\textbf{#1}};
\vspace{0.1cm}
}

\newcommand{\sacado}[1]{ % Exercice directement dans la page
\vspace{0.2cm}
\tikz\node[rounded corners=2pt, fill=violet]{\color{white}\textbf{#1}};
}
 
 
\newcommand{\savoir}[1]{
  {\bf Autre savoir-faire ciblé : } 
  #1
  \medskip
}

\newcommand{\savoirs}[1]{
  %  {\bf Autres savoir-faire ciblés : }
 \vspace{-1em}
  \textcolor{bleusad}{\it \begin{itemize}
      #1
  \end{itemize}
  \centerline{\rule[0.5em]{5cm}{0.5pt}}}
  \medskip
}


\newcommand{\entreexo}{ % entre les exos}
  \medskip
  \centerline{\hrule{5cm}{1pt}}
  \medskip
}

\setlength{\parindent}{0cm}
 

\usepackage{multicol}

%%%%% Pour faire des boites
\usepackage[tikz]{bclogo}
\usepackage{bclogo}
\usepackage{framed}
\usepackage[skins]{tcolorbox}
\tcbuselibrary{breakable}
\tcbuselibrary{skins}
\usetikzlibrary{quotes,babel,arrows.meta,shadows,decorations.pathmorphing,decorations.markings,patterns}
\usepackage{tikzpagenodes}
\usetikzlibrary{plotmarks}

\newcounter{cpt}

\newenvironment{GeneriqueT}[2]{%
\medskip \begin{tcolorbox}[widget,colback=white ,colframe=black,
title= \stepcounter{cpt}  #1  \thecpt \;: #2]}
{%
\end{tcolorbox}\par}


\usepackage{multido}
\newcommand{\point}[1]{\vspace{0.1cm}\multido{}{#1}{ \dotfill \medskip \endgraf}}
\newcommand{\ligne}[1]{\vspace{0.1cm}\multido{}{#1}{ {\color{cqcqcq}\hrulefill} \medskip \endgraf}}



\newcounter{exo}
\newcounter{cptr}




\renewenvironment{leftbar}[1][\hsize]
{% 
    \def\FrameCommand
    {%
       {\color{sacado_purple}\vrule width 1pt}%
        \hspace{0pt}%must no space.
        \fboxsep=\FrameSep%\colorbox{yellow}%
    }%
    \MakeFramed{\hsize#1\advance\hsize-\width\FrameRestore}%
  }
{\endMakeFramed}


\newenvironment{leftbarPink}[1][\hsize]
{% 
    \def\FrameCommand
    {%
       {\color{sacado_pink}\vrule width 1pt}%
        \hspace{0pt}%must no space.
        \fboxsep=\FrameSep%\colorbox{yellow}%
    }%
    \MakeFramed{\hsize#1\advance\hsize-\width\FrameRestore}%
  }
{\endMakeFramed}


\newcommand{\headerGeneral}[1]{ % intitulé, couleur, qrcode
\begin{tikzpicture}[remember picture,overlay]
\coordinate(NO) at (-1.5,0);
\coordinate(SW) at (22,1);
\node[text width=6.2cm,font=\small] at (20.7,-0.04) { \color{white}{#1}  }; 
\end{tikzpicture}
}


%%%%%% Environnement "pageCours" %%%%%%
\newenvironment{pageIntro}{
\rhead{}
\lhead{}%
}
{\newpage} 

%%%%%% Environnement "pageCours" %%%%%%
\newenvironment{pageCours}{
\rhead{\includegraphics[scale=1]{/var/www/sacado/ressources/tex/pageCours.jpg} }
\lhead{}%
\headerGeneral{p/1234}
}
{\newpage}

%%%%%% Environnement "pageAD" %%%%%%
\newenvironment{pageAd}{
\rhead{\includegraphics[scale=1]{/var/www/sacado/ressources/tex/pageAd.jpg} }
\lhead{}%
\headerGeneral{p/1234}
}
{\newpage}

%%%%%% Environnement "pageParcoursu" %%%%%%
\newenvironment{pageParcoursu}{

\rhead{\includegraphics[scale=1]{/var/www/sacado/ressources/tex/pageParcoursu.jpg} }
\lhead{}%
\headerGeneral{p/1234}
}
{\newpage}
%%%%%% Environnement "pageParcoursd" %%%%%%
\newenvironment{pageParcoursd}{
\rhead{\includegraphics[scale=1]{/var/www/sacado/ressources/tex/pageParcoursd.jpg} }
\lhead{}%
\headerGeneral{p/1234}
}
{\newpage}

%%%%%% Environnement "pageParcourst" %%%%%%
\newenvironment{pageParcourst}{
\rhead{\includegraphics[scale=1]{/var/www/sacado/ressources/tex/pageParcourst.jpg} }
\lhead{}%
\headerGeneral{p/1234}
}
{\newpage}
 
 

%%%%%% Environnement "pageAuto" %%%%%%
\newenvironment{pageAuto}{
\rhead{\includegraphics[scale=1]{/var/www/sacado/ressources/tex/pageAuto.jpg} }
\lhead{}%
\headerGeneral{p/1234}
}
{\newpage}

%%%%%% Environnement "pageAlgo" %%%%%%
\newenvironment{pageAlgo}{
\rhead{\includegraphics[scale=1]{/var/www/sacado/ressources/tex/pageAlgo.jpg} }
\lhead{}%
\headerGeneral{p/1234}
}
{\newpage}
 
 

%%%%%% Numéro de page %%%%%%
\fancyfoot[L]{\colorbox{sacado_purple!70}{\color{white}\thepage}}
\fancyfoot[C]{}

%%%%%% Sections et numéros de sections %%%%%%
 
\renewcommand{\thesection}{\arabic{section}}



\titleformat{\section}{}{%
\hspace{-0.15cm}{%
 \tikz[baseline=(a.base)]\node[draw,circle,inner sep=2pt, line width=0mm, outer sep=2pt,fill=sacado_purple!70](a){\Large {\color{white}\thesection}  };
 }}{1em}{\bf \Large {\color{sacado_purple}#1}}
  
  
\renewcommand{\thesubsection}{\arabic{subsection}}

\titleformat{\subsection}{}{%
\hspace{-0.15cm}{%
 \tikz[baseline=(a.base)]\node[draw,circle,inner sep=2pt, outer sep=2pt,fill=sacado_purple!70](a){\large {\color{white}\thesubsection}  };
 }}{1em}{\bf \large {\color{sacado_purple}#1}}




\makeatletter
\newenvironment{TraitV}[1]{%
% #1 couleur du trait (par défaut CouleurA)
% #2 largeur du trait
% #3 distance entre le trait et le texte
\def\FrameCommand{{\color{#1}\vrule width 2pt}
\hspace{1em}}\MakeFramed {\advance\hsize-\width}}%
{\endMakeFramed}
\makeatother

%----------------------------------------
%
%   Définitions des environnements de Définitions, propriétés...
%
%----------------------------------------
\newenvironment{Def}{%
\begin{tcolorbox}[colback=sacado_blue_light,colframe=sacado_blue_light,fonttitle=\bfseries,
 boxsep=5pt,left=5pt,right=5pt,top=2pt,bottom=2pt,
colbacktitle=sacado_blue,  enhanced,outer arc=0pt, attach boxed title to top left={xshift=-2mm, yshift=-3mm},
title= \textbf{$\searrow$}\; Définition. ]\vspace{3mm}
}%
{%
\end{tcolorbox}\par} 



\newenvironment{Th}{%
\begin{tcolorbox}[colback=sacado_yellow_light,colframe=sacado_yellow_light,fonttitle=\bfseries,
 boxsep=5pt,left=5pt,right=5pt,top=2pt,bottom=2pt,
colbacktitle=sacado_pink,  enhanced,outer arc=0pt, attach boxed title to top left={xshift=-2mm, yshift=-3mm},
title= \textbf{$\searrow$}\; Théorème. ]\vspace{3mm}
}%
{%
\end{tcolorbox}\par} 


\newenvironment{Pp}{%
\begin{tcolorbox}[colback=sacado_yellow_light,colframe=sacado_yellow_light,fonttitle=\bfseries,
 boxsep=5pt,left=5pt,right=5pt,top=2pt,bottom=2pt,
colbacktitle=sacado_pink,  enhanced,outer arc=0pt, attach boxed title to top left={xshift=-2mm, yshift=-3mm},
title= \textbf{$\searrow$}\; Propriété. ]\vspace{3mm}
}%
{%
\end{tcolorbox}\par} 


\newenvironment{Att}{%
\begin{tcolorbox}[colback=sacado_pink_light,colframe=sacado_pink_light,fonttitle=\bfseries,
 boxsep=5pt,left=5pt,right=5pt,top=2pt,bottom=2pt,
colbacktitle=sacado_pink,  enhanced,outer arc=0pt, attach boxed title to top left={xshift=-2mm, yshift=-3mm},
title= \textbf{$\searrow$}\; Attention. ]\vspace{3mm}\begin{leftbarPink}}%
{\end{leftbarPink}%
\end{tcolorbox}\par}



\newenvironment{Rq}{%
\begin{tcolorbox}[colback=sacado_yellow_light,colframe=sacado_yellow_light,fonttitle=\bfseries,
 boxsep=5pt,left=5pt,right=5pt,top=2pt,bottom=2pt,
colbacktitle=sacado_pink,  enhanced,outer arc=0pt, attach boxed title to top left={xshift=-2mm, yshift=-3mm},
title= \textbf{$\searrow$}\; Propriété.  ]\vspace{3mm}
}%
{%
\end{tcolorbox}\par} 



\newenvironment{Rg}{%
\begin{tcolorbox}[colback=sacado_yellow_light,colframe=sacado_yellow_light,fonttitle=\bfseries,
 boxsep=5pt,left=5pt,right=5pt,top=2pt,bottom=2pt,
colbacktitle=sacado_pink,  enhanced,outer arc=0pt, attach boxed title to top left={xshift=-2mm, yshift=-3mm},
title= \textbf{$\searrow$}\; Règle. ]\vspace{3mm}
}%
{%
\end{tcolorbox}\par} %%%%%%%%%%%%% Définitions
%%%%%%%%%%%%%  

 
\newenvironment{lBar} {%
\begin{leftbar}
}%
{\end{leftbar}%
}

\newenvironment{Bloc}{%
\hspace{1mm}
\begin{minipage}{0.99\linewidth}
}%
{\end{minipage}%
}





\newenvironment{lBarPink} {%
\begin{leftbarPink}
}%
{\end{leftbarPink}%
}



\newenvironment{Ex}{%
\hspace{1mm} 
\begin{minipage}{0.97\linewidth} \vspace{1mm}
}%
{\vspace{1mm} \end{minipage} %
}



\newtcolorbox{Mt}{%
enhanced,
breakable,
sharp corners,arc=0pt,outer arc=0pt,
boxrule=2pt,boxsep=0pt,top=4pt,left=4pt,right=4pt,bottom=4pt,middle=0pt,
colback=white, 
colframe=white, 
pad at break=0pt,bottomrule at break=0pt,toprule at break=0pt,
borderline east={1pt}{-0.25pt}{sacado_green,dotted},
borderline west={1pt}{-0.25pt}{sacado_green,dotted},
borderline south={1pt}{-0.25pt}{sacado_green,dotted},
borderline north={1pt}{-0.25pt}{sacado_green,dotted},
fontupper={\color{sacado_green}}
        \fcolorbox{sacado_green}{sacado_green}{\color{white}{\small MÉTHODE}}
        \color{sacado_green}\vspace{1mm}
}


\newtcolorbox{ExR}{%
enhanced,
breakable,
sharp corners,arc=0pt,outer arc=0pt,
boxrule=2pt,boxsep=0pt,top=4pt,left=4pt,right=4pt,bottom=4pt,middle=0pt,
colback=white, 
colframe=white, 
pad at break=0pt,bottomrule at break=0pt,toprule at break=0pt,
borderline east={1pt}{-0.25pt}{sacado_green,dotted},
borderline west={1pt}{-0.25pt}{sacado_green,dotted},
borderline south={1pt}{-0.25pt}{sacado_green,dotted},
borderline north={1pt}{-0.25pt}{sacado_green,dotted},
fontupper={\color{sacado_green}}
        \fcolorbox{sacado_green}{sacado_green}{\color{white}{\small EXERCICE RÉSOLU}}
        \color{sacado_green}\vspace{1mm}
}



%%%%%%%%%%%%% ExoCad 7 paramètres : Compétences, qrcode , calculatrice, python, scratch, tableur, annales
 
\newenvironment{Exo}{% code avant
 
\stepcounter{exo}

 
 }

%%%%%%%%%%%%%%%%%%%%%%%%%%%%%%%%%%%%%%%%%%%%%%%%%%%%%%%%%%%%%%%%%%%%%%%%%%%%%%%%%%%%%%%%%%%%%%%%%%%%%%%%%%%%%%%%%%%%%%
%%%%%%%%%%%%%%%%%%%%%%%%%%%%%%%%%%%%%%%%%%%%%%%%%%%%%%%%%%%%%%%%%%%%%%%%%%%%%%%%%%%%%%%%%%%%%%%%%%%%%%%%%%%%%%%%%%%%%%
%%%%%%%%%%%%%%%%  Exercices   sans contours                            %%%%%%%%%%%%%%%%%%%%%%%%%%%%%%%%%%%%%%%%%%%%%%%
%%%%%%%%%%%%%%%%%%%%%%%%%%%%%%%%%%%%%%%%%%%%%%%%%%%%%%%%%%%%%%%%%%%%%%%%%%%%%%%%%%%%%%%%%%%%%%%%%%%%%%%%%%%%%%%%%%%%%%
%%%%%%%%%%%%%%%%%%%%%%%%%%%%%%%%%%%%%%%%%%%%%%%%%%%%%%%%%%%%%%%%%%%%%%%%%%%%%%%%%%%%%%%%%%%%%%%%%%%%%%%%%%%%%%%%%%%%%%


\newcommand{\Sf}[1]{ \vspace{0.1cm}
{\color{sacado_purple}{\Large \textbf{#1}}  } 
} 



\newcommand{\Sfe}[1]{ \vspace{0.1cm}
{\color{sacado_blue}{\Large \textbf{#1}}  } 
} 
 


\usepackage{makeidx}
\makeindex


 